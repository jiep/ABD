\documentclass[12pt,a4paper,twoside,openright,titlepage,final]{article}
\usepackage{fontspec}
\usepackage{amsmath}
\usepackage{amsfonts}
\usepackage{amssymb}
\usepackage{makeidx}
\usepackage{graphicx}
\usepackage[hidelinks,unicode=true]{hyperref}
\usepackage[spanish,es-nodecimaldot,es-lcroman,es-tabla,es-noshorthands]{babel}
\usepackage[left=3cm,right=2cm, bottom=4cm]{geometry}
\usepackage{natbib}
\usepackage{microtype}
\usepackage{ifdraft}
\usepackage{verbatim}
\usepackage[obeyDraft]{todonotes}
\ifdraft{
	\usepackage{draftwatermark}
	\SetWatermarkText{BORRADOR}
	\SetWatermarkScale{0.7}
	\SetWatermarkColor{red}
}{}
\usepackage{booktabs}
\usepackage{longtable}
\usepackage{calc}
\usepackage{array}
\usepackage{caption}
\usepackage{subfigure}
\usepackage{footnote}
\usepackage{url}
\usepackage{tikz}

%\setsansfont[Ligatures=TeX]{texgyreadventor}
%\setmainfont[Ligatures=TeX]{texgyrepagella}

\input{portada}

\author{José Ignacio Escribano}

\title{Práctica final}

\setlength{\parindent}{0pt}

\begin{document}

\pagenumbering{alph}
\setcounter{page}{1}

\portada{Práctica final}{Análisis de Big Data}{Análisis de una comunidad virtual \\ de Stack Exchange}{José Ignacio Escribano}{Móstoles}

\listoftables
\thispagestyle{empty}
\newpage

\tableofcontents
\thispagestyle{empty}
\newpage


\pagenumbering{arabic}
\setcounter{page}{1}

\section{Introducción}

En esta práctica final resolveremos una serie de cuestiones relativas a los usuarios que han participado en la comunidad \url{http://scifi.stackexchange.com}, dedicada a resolver preguntas sobre temas de ciencia ficción y fantasía.\\

El archivo \texttt{users.xml} contiene información sobre el perfil de un grupo del total de usuarios de la comunidad que han participado en ella, hasta marzo de 2015.\\

Utilizaremos Apache Spark para realizar el análisis de este archivo, utilizando un DataFrames.

\section{Carga de los datos}

Para cargar los datos, usamos el siguiente comando:

\begin{verbatim}
df = sqlContext.read.format('com.databricks.spark.xml').options(rowTag='row').load('users.xml')
\end{verbatim}

Este creará un DataFrame a partir del fichero \texttt{users.xml}, que se leerá como archivo \texttt{xml}, y se guardará los hijos de un elemento \texttt{row} como una columna, quedando como una tabla.

\section{Ejercicio 1}

Inspeccionamos las primeras filas del archivo para ver cuáles son los elementos de los que consta cada uno de ellos.

\begin{verbatim}
<users>
  <row>
    <Id>-1</Id>
    <Reputation>1</Reputation>
    <CreationDate>2011-01-11T19:19:36.483</CreationDate>
    <DisplayName>Community</DisplayName>
    <LastAccessDate>2011-01-11T19:19:36.483</LastAccessDate>
    <Location>on the server farm</Location>
    <Views>0</Views>
    <UpVotes>2587</UpVotes>
    <DownVotes>3953</DownVotes>
    <AccountId>-1</AccountId>
  </row>
  <row>
    <Id>2</Id>
    <Reputation>101</Reputation>
    <CreationDate>2011-01-11T19:50:40.620</CreationDate>
    <DisplayName>Geoff Dalgas</DisplayName>
    <LastAccessDate>2015-02-05T00:03:28.030</LastAccessDate>
    <Location>Corvallis, OR</Location>
    <Views>21</Views>
    <UpVotes>1</UpVotes>
    <DownVotes>0</DownVotes>
    <Age>38</Age>
    <AccountId>2</AccountId>
  </row>
  ...
</users>
\end{verbatim}

Observamos que tenemos 11 elementos en cada fila:

\begin{itemize}
	\item AccountId: Id de la cuenta
	\item Age: edad del usuario
	\item CreationDate: fecha de creación de la cuenta
	\item DisplayName: nombre de usuario
	\item DownVotes: votos a favor
	\item Id: Id de usuario
	\item LastAccessDate: fecha del último acceso
	\item Location: localización
	\item Reputation: puntos de reputación
	\item UpVotes: votos a favor
	\item Views: visitas a su página de perfil
\end{itemize}

Otra forma de comprobar el número de columnas de nuestro fichero es imprimir el esquema usando el comando

\begin{verbatim}
df.printSchema()
\end{verbatim}

La salida de este comando es

\begin{verbatim}
root
  |-- AccountId: long (nullable = true)
  |-- Age: long (nullable = true)
  |-- CreationDate: string (nullable = true)
  |-- DisplayName: string (nullable = true)
  |-- DownVotes: long (nullable = true)
  |-- Id: long (nullable = true)
  |-- LastAccessDate: string (nullable = true)
  |-- Location: string (nullable = true)
  |-- Reputation: long (nullable = true)
  |-- UpVotes: long (nullable = true)
  |-- Views: long (nullable = true)
\end{verbatim}

Además de las columnas, este comando nos devuelve el tipo de cada variable.\\

Si queremos ver el contenido del fichero, usamos el comando, que muestra las 10 primeras filas del fichero.

\begin{verbatim}
df.show(10)
\end{verbatim}

El comando devuelve la Tabla~\ref{tbl:head}.\\

\begin{table}[htbp!]
	\centering
	\caption{Primeras líneas del fichero \texttt{users.xml}}
	\label{tbl:head}
	\resizebox{\textwidth}{!}{%
		\begin{tabular}{@{}ccccccccccc@{}}
			\toprule
			\textbf{AccountId} & \textbf{Age} & \textbf{CreationDate} & \textbf{DisplayName} & \textbf{DownVotes} & \textbf{Id} & \textbf{LastAccessDate} & \textbf{Location}    & \textbf{Reputation} & \textbf{UpVotes} & \textbf{Views} \\ \midrule
			-1                 & null         & 2011-01-11T19:19:...  & Community            & 3953               & -1          & 2011-01-11T19:19:...    & on the server farm   & 1                   & 2587             & 0              \\
			2                  & 38           & 2011-01-11T19:50:...  & Geoff Dalgas         & 0                  & 2           & 2015-02-05T00:03:...    & Corvallis, OR        & 101                 & 1                & 21             \\
			7598               & 30           & 2011-01-11T19:55:..   & Nick Craver          & 0                  & 3           & 2015-03-01T13:49:...    & Winston-Salem, NC    & 101                 & 3                & 10             \\
			1998               & 29           & 2011-01-11T20:17:...  & Emmett               & 0                  & 4           & 2013-05-06T20:52:...    & San Francisco, CA    & 101                 & 0                & 7              \\
			29738              & null         & 2011-01-11T20:18:...  & Kevin Montrose       & 0                  & 5           & 2015-02-15T05:27:...    & New York City, Ne... & 101                 & 32               & 11             \\
			32917              & 30           & 2011-01-11T20:29:...  & David Fullerton      & 4                  & 6           & 2015-03-06T22:57:...    & New York, NY         & 99                  & 63               & 13             \\
			33603              & 30           & 2011-01-11T20:37:...  & Sorantis             & 0                  & 7           & 2014-07-23T12:00:...    & Sweden               & 146                 & 5                & 15             \\
			51549              & 38           & 2011-01-11T20:37:...  & GAThrawn             & 0                  & 8           & 2015-02-18T23:17:...    & United Kingdom       & 103                 & 19               & 98             \\
			3874               & 48           & 2011-01-11T20:38:...  & Rodger Cooley        & 3                  & 9           & 2015-03-04T19:47:...    & Houston, TX          & 1816                & 211              & 85             \\
			15651              & 42           & 2011-01-11T20:38:...  & MatthewMartin        & 2            & 10          & 2015-03-06T01:30:...    & United States        & 1787                & 78               & 48             \\ \bottomrule
		\end{tabular}%
	}
\end{table}

\section{Ejercicio 2}
A continuación, resolveremos una serie de consultas sobre el fichero XML.

\begin{enumerate}
	\item Número total de registros de usuarios en el fichero.
	
	La consulta que realizamos es la siguiente:
	
	\begin{verbatim}
	totalUsers = df.count()
	print(totalUsers)
	\end{verbatim}
	
	La consulta devuelve 20333. Es decir, tenemos que el fichero tiene 20333 usuarios.
	
	\item Todos los datos disponibles sobre el usuario con \texttt{DisplayName} igual a \texttt{ambient\_memory}.
	
	La consulta que realizamos es la siguiente:
	
	\begin{verbatim}
	df.filter(df["DisplayName"] == "ambient_memory").show()
	\end{verbatim}
	
	La consulta busca las filas que tienen \texttt{ambient\_memory} como \texttt{DisplayName} y mostramos el resultado en forma tabular. La Tabla~\ref{tbl:ambient_memory} muestra los resultados.
	
	\begin{table}[htbp!]
		\centering
		\caption{Información del usuario \texttt{ambient\_memory}}
		\label{tbl:ambient_memory}
		\resizebox{\textwidth}{!}{%
			\begin{tabular}{@{}ccccccccccc@{}}
				\toprule
				\textbf{AccountId} & \textbf{Age} & \textbf{CreationDate} & \textbf{DisplayName} & \textbf{DownVotes} & \textbf{Id} & \textbf{LastAccessDate} & \textbf{Location}    & \textbf{Reputation} & \textbf{UpVotes} & \textbf{Views} \\ \midrule
				4529331            & null         & 2014-06-20T12:01:...  & ambient\_memory      & 0                  & 28494       & 2014-06-20T12:01:...    & New York, United ... & 1                   & 0                & 0              \\ \bottomrule
			\end{tabular}%
		}
	\end{table}
	
	\item Los 10 usuarios con mayor reputación.\\
	
	La consulta es la siguiente:
	
	\begin{verbatim}
	df.sort(df.Reputation.desc()).limit(10).show()
	
	''' O de forma equivalente '''
	
	df.sort("Reputation", ascending=False).limit(10).show()
	\end{verbatim}
	
	La consulta ordena la columna \texttt{Reputation} de forma descendente y selecciona las 10 primeras filas y el resultado se muestra en forma tabular. La Tabla~\ref{tbl:reputation} muestra los resultados de esta consulta.
	
	\begin{table}[htbp!]
		\centering
		\caption{Los 10 usuarios con más reputación}
		\label{tbl:reputation}
		\resizebox{\textwidth}{!}{%
			\begin{tabular}{@{}ccccccccccc@{}}
				\toprule
				\textbf{AccountId} & \textbf{Age} & \textbf{CreationDate} & \textbf{DisplayName} & \textbf{DownVotes} & \textbf{Id} & \textbf{LastAccessDate} & \textbf{Location}    & \textbf{Reputation} & \textbf{UpVotes} & \textbf{Views} \\ \midrule
				41067              & 43           & 2011-02-28T04:00:...  & DVK                  & 1806               & 1806        & 2015-03-08T02:21:...    & New York, NY         & 148259              & 5503             & 6224           \\
				893673             & null         & 2011-09-06T19:47:...  & Thaddeus             & 41                 & 2765        & 2015-03-07T21:21:...    & Hayward, CA          & 118763              & 1934             & 3101           \\
				3776439            & null         & 2013-12-26T15:54:...  & Richard              & 2288               & 20774       & 2015-03-08T02:36:...    & UK                   & 92783               & 3878             & 7577           \\
				1057622            & null         & 2011-11-22T14:47:...  & Slytherincess        & 179                & 3500        & 2015-03-08T00:08:...    & Azkaban              & 67739               & 841              & 3562           \\
				12203              & 32           & 2011-02-01T22:00:...  & Jeff                 & 44                 & 656         & 2015-03-08T02:28:...    & Cincinnati, OH       & 62833               & 2230             & 1362           \\
				486482             & 44           & 2012-09-10T13:52:...  & Darth Satan          & 192                & 8719        & 2015-03-07T21:49:...    & null                 & 57681               & 910              & 1515           \\
				375283             & null         & 2011-03-07T02:09:...  & Keen                 & 1618               & 1027        & 2015-03-08T01:44:...    & null                 & 49731               & 4875             & 2560           \\
				359788             & 53           & 2011-04-28T17:25:...  & Tango                & 7                  & 1693        & 2015-03-06T02:46:...    & Richmond, VA, USA    & 48075               & 1919             & 1517           \\
				102643             & 36           & 2011-01-11T20:56:...  & DavRob60             & 242                & 45          & 2015-03-06T11:41:...    & Salaberry-de-Vall... & 47505               & 3799             & 1671           \\
				1170648|           & null         & 2012-03-06T21:45:...  & phantom42            & 406                & 5184        & 2015-03-08T03:15:...    & Orlando, FL          & 47257               & 2003             & 1771           \\ \bottomrule
			\end{tabular}%
		}
	\end{table}
		
	\item Los usuarios con la fecha de creación más antigua y la más reciente, respectivamente.\\
	
	La consulta es la siguiente:
	
	\begin{verbatim}
	''' Usuario más antiguo '''
	df.sort("CreationDate", ascending=False).limit(1).show()
	
	''' Usuario más reciente '''
	df.sort("CreationDate", ascending=True).limit(1).show()
	\end{verbatim}
	
	La consulta ordena descendentemente (ascentemente) por la columna \texttt{CreationDate}, se selecciona la primera fila y se muestra de forma tabular. La primera consulta devuelve el usuario más antiguo y la segunda, el usuario más reciente.\\
	
	Las Tablas~\ref{tbl:mas_antiguo}~y~\ref{tbl:mas_reciente} muestra los usuarios con fecha de creación de su cuenta más antigua y más reciente, respectivamente.\\  
	
	\begin{table}[htbp!]
		\centering
		\caption{Usuario más antiguo}
		\label{tbl:mas_antiguo}
		\resizebox{\textwidth}{!}{%
			\begin{tabular}{@{}ccccccccccc@{}}
				\toprule
				\textbf{AccountId} & \textbf{Age} & \textbf{CreationDate} & \textbf{DisplayName} & \textbf{DownVotes} & \textbf{Id} & \textbf{LastAccessDate} & \textbf{Location}    & \textbf{Reputation} & \textbf{UpVotes} & \textbf{Views} \\ \midrule
				4529331            & null         & 2014-06-20T12:01:...  & ambient\_memory      & 0                  & 28494       & 2014-06-20T12:01:...    & New York, United ... & 1                   & 0                & 0              \\ \bottomrule
			\end{tabular}%
		}
	\end{table}
	
	
	\begin{table}[htbp!]
		\centering
		\caption{Usuario más reciente}
		\label{tbl:mas_reciente}
		\resizebox{\textwidth}{!}{%
			\begin{tabular}{@{}ccccccccccc@{}}
				\toprule
				\textbf{AccountId} & \textbf{Age} & \textbf{CreationDate} & \textbf{DisplayName} & \textbf{DownVotes} & \textbf{Id} & \textbf{LastAccessDate} & \textbf{Location}  & \textbf{Reputation} & \textbf{UpVotes} & \textbf{Views} \\ \midrule
				-1                 & null         & 2011-01-11T19:19:...  & Community            & 3953               & -1          & 2011-01-11T19:19:...    & on the server farm & 1                   & 2587             & 0              \\ \bottomrule
			\end{tabular}%
		}
	\end{table}
	
	\item El usuario más joven y el más viejo (de los que han indicado una edad válida en el campo \texttt{Age}).\\
	
	La consulta es la siguiente:
	
	\begin{verbatim}
	''' Importamos las funciones `max` y `min` '''
	from pyspark.sql.functions import min, max
	
	''' Calculamos la edad máxima y mínima de la columna `Age` '''
	ages = df.select(min("Age").alias("min"), max("Age").alias("max")).collect()
	\end{verbatim}
	
	La consulta devuelve 
	
	\begin{verbatim}
	[Row(min=14, max=95)]
	\end{verbatim}
	
	Es decir, la edad mínima es 14 y la máxima es 95.\\
	
	Para conocer la información de los usuarios que tienen estas edades, hacemos la siguiente consulta:
	
	\begin{verbatim}
	''' Usuarios con la edad mínima guardada en la variable `ages` '''
	df.filter(df.Age == ages[0].min).show()
	
	''' Usuarios con la edad máxima guardada en la variable `ages` '''
	df.filter(df.Age == ages[0].max).show()
	\end{verbatim}
	
	La consulta muestra los usuarios que tienen la edad mínima o máxima y se muestra de forma tabular. Las Tablas~\ref{tbl:menos_edad}~y~\ref{tbl:mas_edad} muestran a los usuarios más jóvenes y viejos, respectivamente.\\  
	
	\begin{table}[htbp!]
		\centering
		\caption{Usuarios más jovenes}
		\label{tbl:menos_edad}
		\resizebox{\textwidth}{!}{%
			\begin{tabular}{@{}ccccccccccc@{}}
				\toprule
				\textbf{AccountId} & \textbf{Age} & \textbf{CreationDate} & \textbf{DisplayName} & \textbf{DownVotes} & \textbf{Id} & \textbf{LastAccessDate} & \textbf{Location} & \textbf{Reputation} & \textbf{UpVotes} & \textbf{Views} \\ \midrule
				2178726            & 14           & 2013-05-01T20:06:...  & danilka1             & 0                  & 14208       & 2013-05-01T20:06:...    & planet earth      & 101                 & 0                & 0              \\
				198592             & 14           & 2013-08-29T15:11:...  & Ramchandra Apte      & 0                  & 16999       & 2014-12-24T13:08:...    & Bangalore, India  & 103                 & 9                & 3              \\ \bottomrule
			\end{tabular}%
		}
	\end{table}
	
	
\end{enumerate}
\section{Conclusiones}

\newpage

\section{Código Python}

\begin{verbatim}

\end{verbatim}


\end{document} 