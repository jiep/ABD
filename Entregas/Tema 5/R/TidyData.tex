\documentclass[12pt,spanish,]{article}
\usepackage{lmodern}
\usepackage{amssymb,amsmath}
\usepackage{ifxetex,ifluatex}
\usepackage{fixltx2e} % provides \textsubscript
\ifnum 0\ifxetex 1\fi\ifluatex 1\fi=0 % if pdftex
  \usepackage[T1]{fontenc}
  \usepackage[utf8]{inputenc}
\else % if luatex or xelatex
  \ifxetex
    \usepackage{mathspec}
    \usepackage{xltxtra,xunicode}
  \else
    \usepackage{fontspec}
  \fi
  \defaultfontfeatures{Mapping=tex-text,Scale=MatchLowercase}
  \newcommand{\euro}{€}
\fi
% use upquote if available, for straight quotes in verbatim environments
\IfFileExists{upquote.sty}{\usepackage{upquote}}{}
% use microtype if available
\IfFileExists{microtype.sty}{%
\usepackage{microtype}
\UseMicrotypeSet[protrusion]{basicmath} % disable protrusion for tt fonts
}{}
\usepackage[margin=1in]{geometry}
\ifxetex
  \usepackage{polyglossia}
  \setmainlanguage{}
\else
  \usepackage[shorthands=off,spanish]{babel}
\fi
\usepackage{color}
\usepackage{fancyvrb}
\newcommand{\VerbBar}{|}
\newcommand{\VERB}{\Verb[commandchars=\\\{\}]}
\DefineVerbatimEnvironment{Highlighting}{Verbatim}{commandchars=\\\{\}}
% Add ',fontsize=\small' for more characters per line
\usepackage{framed}
\definecolor{shadecolor}{RGB}{248,248,248}
\newenvironment{Shaded}{\begin{snugshade}}{\end{snugshade}}
\newcommand{\KeywordTok}[1]{\textcolor[rgb]{0.13,0.29,0.53}{\textbf{{#1}}}}
\newcommand{\DataTypeTok}[1]{\textcolor[rgb]{0.13,0.29,0.53}{{#1}}}
\newcommand{\DecValTok}[1]{\textcolor[rgb]{0.00,0.00,0.81}{{#1}}}
\newcommand{\BaseNTok}[1]{\textcolor[rgb]{0.00,0.00,0.81}{{#1}}}
\newcommand{\FloatTok}[1]{\textcolor[rgb]{0.00,0.00,0.81}{{#1}}}
\newcommand{\CharTok}[1]{\textcolor[rgb]{0.31,0.60,0.02}{{#1}}}
\newcommand{\StringTok}[1]{\textcolor[rgb]{0.31,0.60,0.02}{{#1}}}
\newcommand{\CommentTok}[1]{\textcolor[rgb]{0.56,0.35,0.01}{\textit{{#1}}}}
\newcommand{\OtherTok}[1]{\textcolor[rgb]{0.56,0.35,0.01}{{#1}}}
\newcommand{\AlertTok}[1]{\textcolor[rgb]{0.94,0.16,0.16}{{#1}}}
\newcommand{\FunctionTok}[1]{\textcolor[rgb]{0.00,0.00,0.00}{{#1}}}
\newcommand{\RegionMarkerTok}[1]{{#1}}
\newcommand{\ErrorTok}[1]{\textbf{{#1}}}
\newcommand{\NormalTok}[1]{{#1}}
\usepackage{graphicx}
\makeatletter
\def\maxwidth{\ifdim\Gin@nat@width>\linewidth\linewidth\else\Gin@nat@width\fi}
\def\maxheight{\ifdim\Gin@nat@height>\textheight\textheight\else\Gin@nat@height\fi}
\makeatother
% Scale images if necessary, so that they will not overflow the page
% margins by default, and it is still possible to overwrite the defaults
% using explicit options in \includegraphics[width, height, ...]{}
\setkeys{Gin}{width=\maxwidth,height=\maxheight,keepaspectratio}
\ifxetex
  \usepackage[setpagesize=false, % page size defined by xetex
              unicode=false, % unicode breaks when used with xetex
              xetex]{hyperref}
\else
  \usepackage[unicode=true]{hyperref}
\fi
\hypersetup{breaklinks=true,
            bookmarks=true,
            pdfauthor={José Ignacio Escribano},
            pdftitle={Tidy Data con R},
            colorlinks=true,
            citecolor=blue,
            urlcolor=blue,
            linkcolor=magenta,
            pdfborder={0 0 0}}
\urlstyle{same}  % don't use monospace font for urls
\setlength{\parindent}{0pt}
\setlength{\parskip}{6pt plus 2pt minus 1pt}
\setlength{\emergencystretch}{3em}  % prevent overfull lines
\setcounter{secnumdepth}{5}

%%% Use protect on footnotes to avoid problems with footnotes in titles
\let\rmarkdownfootnote\footnote%
\def\footnote{\protect\rmarkdownfootnote}

%%% Change title format to be more compact
\usepackage{titling}

% Create subtitle command for use in maketitle
\newcommand{\subtitle}[1]{
  \posttitle{
    \begin{center}\large#1\end{center}
    }
}

\setlength{\droptitle}{-2em}
  \title{Tidy Data con R}
  \pretitle{\vspace{\droptitle}\centering\huge}
  \posttitle{\par}
  \author{José Ignacio Escribano}
  \preauthor{\centering\large\emph}
  \postauthor{\par}
  \predate{\centering\large\emph}
  \postdate{\par}
  \date{10 de abril de 2016}



\begin{document}

\maketitle


{
\hypersetup{linkcolor=black}
\setcounter{tocdepth}{2}
\tableofcontents
}
\section{Introducción}\label{introduccion}

En primer lugar, cargamos los paquetes necesarios para limpiar los
datos.

En caso de que de se muestre el siguiente error

\begin{verbatim}
Error in library(dplyr) : there is no package called ‘dplyr’
Error in library(dplyr) : there is no package called ‘tidyr’
\end{verbatim}

debemos ejecutar el siguiente comando, que instalará los paquetes
anteriores

\begin{verbatim}
install.packages(c("dplyr", "tidyr"))
\end{verbatim}

Una vez instalados los paquetes, volvemos a ejecutar los comandos
anteriores.

Leemos el archivo \texttt{Air\_Quality.csv}, que se encuentra en la
misma carpeta que este documento usando los siguientes comandos.

\begin{Shaded}
\begin{Highlighting}[]
\NormalTok{filename =}\StringTok{ "Air_Quality.csv"}
\NormalTok{data_csv =}\StringTok{ }\KeywordTok{tbl_df}\NormalTok{(}\KeywordTok{read.csv}\NormalTok{(}\DataTypeTok{file =} \NormalTok{filename, }\DataTypeTok{sep =} \StringTok{","}\NormalTok{, }\DataTypeTok{header =} \OtherTok{TRUE}\NormalTok{))}
\end{Highlighting}
\end{Shaded}

\section{Limpieza de los datos}\label{limpieza-de-los-datos}

Ya estamos en disposición para limpiar los datos. Para ello:

\begin{enumerate}
\def\labelenumi{\arabic{enumi}.}
\item
  Seleccionamos de nuestro fichero de datos \texttt{data} las variables
  \texttt{year\_description}, \texttt{geo\_entity\_id},
  \texttt{geo\_type\_name}, \texttt{data\_valuemessage} y
  \texttt{indicator\_id}. Notar que en esta última variable nos servirá
  para filtrar aquellas filas que se corresponden con el identificador
  646, que es el del benceno.
\item
  Filtramos las que contengan el identificador 646, es decir, el del
  benceno.
\item
  Cambiamos el nombre a las columnas.
\item
  Eliminamos la columna \texttt{indicator\_id}.
\end{enumerate}

\begin{Shaded}
\begin{Highlighting}[]
\KeywordTok{attach}\NormalTok{(data_csv)}
\CommentTok{# Paso 1}
\NormalTok{selected_columns =}\StringTok{ }\KeywordTok{select}\NormalTok{(data_csv, }
                          \NormalTok{year_description, }
                          \NormalTok{geo_entity_id, }
                          \NormalTok{geo_type_name, }
                          \NormalTok{data_valuemessage, }
                          \NormalTok{indicator_id)}

\CommentTok{# Paso 2}
\NormalTok{filtered_rows =}\StringTok{ }\KeywordTok{filter}\NormalTok{(selected_columns, indicator_id ==}\StringTok{ }\DecValTok{646}\NormalTok{)}

\CommentTok{# Paso 3}
\NormalTok{renamed_columns =}\StringTok{ }\KeywordTok{rename}\NormalTok{(filtered_rows,  }\DataTypeTok{year=} \NormalTok{year_description, }
                         \DataTypeTok{geo_entity =} \NormalTok{geo_entity_id, }
                         \DataTypeTok{geo_type =} \NormalTok{geo_type_name,}
                         \DataTypeTok{data_value =} \NormalTok{data_valuemessage)}


\CommentTok{# Paso 4}
\NormalTok{tidy_data =}\StringTok{ }\KeywordTok{select}\NormalTok{(renamed_columns, -indicator_id)}
\end{Highlighting}
\end{Shaded}

Nuestro nuevo conjunto de datos tiene el siguiente aspecto:

\begin{Shaded}
\begin{Highlighting}[]
\KeywordTok{head}\NormalTok{(tidy_data)}
\end{Highlighting}
\end{Shaded}

\begin{verbatim}
## Source: local data frame [6 x 4]
## 
##     year geo_entity geo_type data_value
##   (fctr)      (int)   (fctr)      (dbl)
## 1   2005          1  Borough        2.8
## 2   2005          2  Borough        2.8
## 3   2005          3  Borough        4.7
## 4   2005          4  Borough        1.9
## 5   2005          5  Borough        1.6
## 6   2005          1 Citywide        2.9
\end{verbatim}

Comprobamos el tamaño de este nuevo conunto de datos:

\begin{Shaded}
\begin{Highlighting}[]
\KeywordTok{dim}\NormalTok{(tidy_data)}
\end{Highlighting}
\end{Shaded}

\begin{verbatim}
## [1] 48  4
\end{verbatim}

Es decir, tenemos unos datos con 48 filas y 4 columnas.

Otra forma de hacer lo anterior es usando el operador
\texttt{\%\textgreater{}\%}

\begin{Shaded}
\begin{Highlighting}[]
\NormalTok{tidy_data2=data_csv %>%}\StringTok{ }
\StringTok{  }
\StringTok{  }\KeywordTok{select}\NormalTok{(year_description, }
         \NormalTok{geo_entity_id, }
         \NormalTok{geo_type_name, }
         \NormalTok{data_valuemessage, }
         \NormalTok{indicator_id) %>%}
\StringTok{  }\KeywordTok{filter}\NormalTok{(indicator_id ==}\StringTok{ }\DecValTok{646}\NormalTok{)  %>%}\StringTok{ }

\StringTok{  }\KeywordTok{rename}\NormalTok{(}\DataTypeTok{year=} \NormalTok{year_description, }
         \DataTypeTok{geo_entity =} \NormalTok{geo_entity_id, }
         \DataTypeTok{geo_type =} \NormalTok{geo_type_name,}
         \DataTypeTok{data_value =} \NormalTok{data_valuemessage) %>%}

\StringTok{  }\KeywordTok{select}\NormalTok{(-indicator_id)}
\end{Highlighting}
\end{Shaded}

Comprobamos que las dimensiones coinciden con lo obtenido anteriormente.

\begin{Shaded}
\begin{Highlighting}[]
\KeywordTok{dim}\NormalTok{(tidy_data2)}
\end{Highlighting}
\end{Shaded}

\begin{verbatim}
## [1] 48  4
\end{verbatim}

Y, por último comprobamos que los vectores son idénticos, es decir, si
todas las posiciones son iguales.

\begin{Shaded}
\begin{Highlighting}[]
\NormalTok{equal_index =}\StringTok{ }\KeywordTok{sum}\NormalTok{(tidy_data ==}\StringTok{ }\NormalTok{tidy_data2)}

\KeywordTok{cat}\NormalTok{(}\StringTok{"El número de índices iguales es"}\NormalTok{, equal_index)}
\end{Highlighting}
\end{Shaded}

\begin{verbatim}
## El número de índices iguales es 192
\end{verbatim}

Esto nos devuelve 192 (48*4) índices que concuerdan (todos los valores
son iguales posición a posición), por lo que ambos resultados son
idénticos.

\section{Ejercicio 1}\label{ejercicio-1}

Obtenemos algunas medidas estadísticas para la columna
\texttt{data\_value}.

\begin{Shaded}
\begin{Highlighting}[]
\CommentTok{# Guardamos los datos en la variable data_value}
\NormalTok{data_value =}\StringTok{ }\NormalTok{tidy_data$data_value}

\KeywordTok{cat}\NormalTok{(}\StringTok{"La media es"}\NormalTok{, }\KeywordTok{mean}\NormalTok{(data_value), }\StringTok{"y la media es"}\NormalTok{, }\KeywordTok{median}\NormalTok{(data_value))}
\end{Highlighting}
\end{Shaded}

\begin{verbatim}
## La media es 2.910417 y la media es 2.75
\end{verbatim}

\begin{Shaded}
\begin{Highlighting}[]
\KeywordTok{cat}\NormalTok{(}\StringTok{"La desviación típica es"}\NormalTok{, }\KeywordTok{sd}\NormalTok{(data_value))}
\end{Highlighting}
\end{Shaded}

\begin{verbatim}
## La desviación típica es 1.166599
\end{verbatim}

\begin{Shaded}
\begin{Highlighting}[]
\KeywordTok{cat}\NormalTok{(}\StringTok{"El mínimo es"}\NormalTok{, }\KeywordTok{min}\NormalTok{(data_value), }\StringTok{"y el máximo es"}\NormalTok{, }\KeywordTok{max}\NormalTok{(data_value))}
\end{Highlighting}
\end{Shaded}

\begin{verbatim}
## El mínimo es 1.1 y el máximo es 6.3
\end{verbatim}

\begin{Shaded}
\begin{Highlighting}[]
\KeywordTok{cat}\NormalTok{(}\StringTok{"El primer cuartil es"}\NormalTok{, }\KeywordTok{quantile}\NormalTok{(data_value, }\FloatTok{0.25}\NormalTok{)[[}\DecValTok{1}\NormalTok{]], }
    \StringTok{"y el tercer cuartil es"}\NormalTok{, }\KeywordTok{quantile}\NormalTok{(data_value, }\FloatTok{0.75}\NormalTok{)[[}\DecValTok{1}\NormalTok{]])}
\end{Highlighting}
\end{Shaded}

\begin{verbatim}
## El primer cuartil es 1.975 y el tercer cuartil es 3.7
\end{verbatim}

\section{Ejercicio 2}\label{ejercicio-2}

Representamos los datos de la variable \texttt{data\_value} como un
histograma.

\begin{Shaded}
\begin{Highlighting}[]
\NormalTok{title_hist =}\StringTok{ "Concentración media de particulas de benceno}\CharTok{\textbackslash{}n}\StringTok{ en la ciudad de Nueva York"}
\NormalTok{ylabel_hist =}\StringTok{ "Frecuencia"}
\NormalTok{xlabel_hist =}\StringTok{ ""}
\KeywordTok{hist}\NormalTok{(data_value, }
     \DataTypeTok{main =} \NormalTok{title_hist, }
     \DataTypeTok{xlab =} \NormalTok{xlabel_hist, }
     \DataTypeTok{ylab =} \NormalTok{ylabel_hist,}
     \DataTypeTok{col =} \StringTok{"blue"}\NormalTok{)}
\end{Highlighting}
\end{Shaded}

\includegraphics{TidyData_files/figure-latex/unnamed-chunk-10-1.pdf}

\section{Ejercicio 3}\label{ejercicio-3}

Dibujamos la función de densidad y de distribución de la variable
\texttt{data\_value}.

\begin{Shaded}
\begin{Highlighting}[]
\CommentTok{# Dividimos la pantalla en dos}
\KeywordTok{par}\NormalTok{(}\DataTypeTok{mfrow=}\KeywordTok{c}\NormalTok{(}\DecValTok{1}\NormalTok{,}\DecValTok{2}\NormalTok{))}

\CommentTok{# Función de densidad}
\NormalTok{title_density =}\StringTok{ "Función de densidad"}
\NormalTok{ylabel_density =}\StringTok{ "Densidad"}
\NormalTok{xlabel_density =}\StringTok{ ""}

\KeywordTok{plot}\NormalTok{(}\KeywordTok{density}\NormalTok{(data_value),}
     \DataTypeTok{main =} \NormalTok{title_density,}
     \DataTypeTok{xlab =} \NormalTok{xlabel_density,}
     \DataTypeTok{ylab =} \NormalTok{ylabel_density)}

\CommentTok{# Función de probabilidad}
\NormalTok{title_cdf =}\StringTok{ "Función de probabilidad"}
\NormalTok{ylabel_cdf =}\StringTok{ "Probabilidad acumulada"}
\NormalTok{xlabel_cdf =}\StringTok{ ""}
\KeywordTok{plot}\NormalTok{(}\KeywordTok{ecdf}\NormalTok{(data_value),}
     \DataTypeTok{main =} \NormalTok{title_cdf,}
     \DataTypeTok{xlab =} \NormalTok{xlabel_cdf,}
     \DataTypeTok{ylab =} \NormalTok{ylabel_cdf)}
\end{Highlighting}
\end{Shaded}

\includegraphics{TidyData_files/figure-latex/unnamed-chunk-11-1.pdf}

\end{document}
