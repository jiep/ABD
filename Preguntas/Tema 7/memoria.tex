\documentclass[12pt,a4paper,twoside,openright,titlepage,final]{article}
\usepackage{fontspec}
\usepackage{amsmath}
\usepackage{amsfonts}
\usepackage{amssymb}
\usepackage{makeidx}
\usepackage{graphicx}
\usepackage[hidelinks,unicode=true]{hyperref}
\usepackage[spanish,es-nodecimaldot,es-lcroman,es-tabla,es-noshorthands]{babel}
\usepackage[left=3cm,right=2cm, bottom=4cm]{geometry}
\usepackage{natbib}
\usepackage{microtype}
\usepackage{ifdraft}
\usepackage{verbatim}
\usepackage[nottoc]{tocbibind}
\usepackage{pdflscape}
\usepackage{fancyvrb}
\usepackage[obeyDraft]{todonotes}
\ifdraft{
	\usepackage{draftwatermark}
	\SetWatermarkText{BORRADOR}
	\SetWatermarkScale{0.7}
	\SetWatermarkColor{red}
}{}
\usepackage{booktabs}
\usepackage{longtable}
\usepackage{calc}
\usepackage{array}
\usepackage{caption}
\usepackage{subfigure}
\usepackage{footnote}
\usepackage{url}
\usepackage[titletoc]{appendix}

\setsansfont[Ligatures=TeX]{texgyreadventor}
\setmainfont[Ligatures=TeX]{texgyrepagella}
\setmonofont{FreeMono}

\usetikzlibrary{decorations.pathreplacing}

\input{portada}

\author{José Ignacio Escribano}

\title{}
\setlength{\parindent}{0pt}

\begin{document}

\pagenumbering{alph}
\setcounter{page}{1}

\portada{Foro de preguntas}{Análisis de Big Data}{Preguntas Tema 7}{José Ignacio Escribano}{Móstoles}

\tableofcontents
\thispagestyle{empty}
\newpage

\pagenumbering{arabic}
\setcounter{page}{1}


\section{Preguntas}

\subsection{A pesar de su nombre, tomado del villano de la saga Harry Potter, Voldemort es una conocida base de datos NoSQL. Responda a estas preguntas: ¿En qué empresa surgió? ¿En cual de las cuatro familiar NoSQL presentadas en teoría se puede encuadrar? ¿Cuáles son sus principales ventajas e inconvenientes?}



Fuentes:\\
\begin{verbatim}
[1] 
\end{verbatim}

\subsection{Busque información sobre la primera base de datos (por fecha de lanzamiento) con arquitectura clave-valor de la familia NoSQL. ¿Dónde surgió, en una empresa o como proyecto de software libre? ¿Qué factores o necesidades determinaron su aparición?}

Fuentes:\\
\begin{verbatim}
[1]
\end{verbatim}

\end{document} 